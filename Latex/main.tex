%%%%%%%%%%%%%%%%%%%%%%%%%%%%%%%%%%%%%%%%%
% Lachaise Assignment
% LaTeX Template
% Version 1.0 (26/6/2018)
%
% This template originates from:
% http://www.LaTeXTemplates.com
%
% Authors:
% Marion Lachaise & François Févotte
% Vel (vel@LaTeXTemplates.com)
%
% License:
% CC BY-NC-SA 3.0 (http://creativecommons.org/licenses/by-nc-sa/3.0/)
% 
%%%%%%%%%%%%%%%%%%%%%%%%%%%%%%%%%%%%%%%%%

%----------------------------------------------------------------------------------------
%	PACKAGES AND OTHER DOCUMENT CONFIGURATIONS
%----------------------------------------------------------------------------------------

\documentclass{article}

\input{structure.tex} % Include the file specifying the document structure and custom commands

%----------------------------------------------------------------------------------------
%	ASSIGNMENT INFORMATION
%----------------------------------------------------------------------------------------

\title{Introduction to Artificial Intelligence: Assignment \#1\\Fast Trajectory Replanning} % Title of the assignment

\author{Yuyang Chen. -- yc791\\ Simiao fan -- sf578\\ Zhaohan Yan -- zy134} % Author name and email address

\date{Rutgers University --- \today} % University, school and/or department name(s) and a date

%----------------------------------------------------------------------------------------

\begin{document}

\maketitle % Print the title

%----------------------------------------------------------------------------------------
%	INTRODUCTION
%----------------------------------------------------------------------------------------

\section*{Introduction} % Unnumbered section

\begin{info} % Information block
	This document is Homework 1 for 198:440 Introduction to Artificial Intelligence at Rutgers University Spring 2019.
\end{info}

%----------------------------------------------------------------------------------------
%	Part 1 - Understanding the methods
%----------------------------------------------------------------------------------------

\section{Part 1 - Understanding the methods} % Numbered section

Read the chapter in your textbook on uninformed and informed
(heuristic) search and then read the project description again. Make sure that you understand A* and the concepts of
admissible and consistent h-values.

%------------------------------------------------

\subsection{}

% Numbered question, with subquestions in an enumerate environment
\begin{question}
	Explain in your report why the first move of the agent for the example search problem from Figure 8 is to the east rather
than the north given that the agent does not know initially which cells are blocked.

\end{question}

Answer: \newline
\par When the agent begins in the maze program, it does not initially know which cells are blocked. It has to survey its surroundings as it takes each proceeding step. In Figure 8, the agent takes a step to the east rather than north because A* would calculate that it would take a shorter distance to the goal to go east, rather than first north and then east. Our agent can only see which cells are blocked at each subsequent step, so at its initial position, it cannot see that the path directly east is blocked both to the right and from above. Only once it moves east will it realize that it is not the correct path, and then move back to go north.
	
%------------------------------------------------

\subsection{}

% Numbered question, with subquestions in an enumerate environment
\begin{question}
	This project argues that the agent is guaranteed to reach the target if it is not separated from it by blocked cells. Give a
convincing argument that the agent in finite gridworlds indeed either reaches the target or discovers that this is impossible
in finite time. Prove that the number of moves of the agent until it reaches the target or discovers that this is impossible is
bounded from above by the number of unblocked cells squared.

\end{question}

Answer: \newline
\par In this gridworld, the agent is guaranteed to reach the target of the maze (goal) or discover it is impossible to as A* search leads our agent through every possible cell in the grid to the goal. If the agent detect that the goal state is blocked on all sides by obstacles such as the grid edge or blocked cells, in the next A* search, the open list would be empty before the the algorithm pushing the goal stage node into the open list, and therefore we know we can’t reach to the goal stage. Otherwise, the agent will keep moving along each cell until it reaches goal.

Use induction to prove: "The number of moves of the agent until it reaches the target or discovers that this is impossible is bounded from above by the number of unblocked cells squared. Let n be the total number of cells, let m be the number of unblocked cells, and let k be the
total number of moves the agent took

Base case: n = 2, when there is only start state and goal state. If the goal state is blocked,
then the total move would be k = 0, m = 1.

$$k < m^2  --- check$$

If the goal state is unblocked, then m = 2, k = 1, because we directly move from start to goal with a single step. 

$$k < m^2 --- check$$

Inductive step: For n > 2, Suppose 

$$k < m'^2$$
 
is true for all  m' belong [1, m),need to prove $$k < m^2$$ is also true. We know that for each unblocked cell, the agent would at most visit it twice (the first time follow the path to the goal, and later come back with the updated blocked information). Therefore, $k <= 2m.$ 

$$(\frac{2m}{m^2}) = (\frac{2}{m})$$, 

since $m > m' >= 1,$ therefore $m >= 2,$ and further 

$$(\frac{2}{m}) <= 1$$. 

Thus k'/2m <= 1. Therefore $k < m^2$ hold true.
Based on the base step and inductive step, we know $k < m^2$ for all m >= 1.



%----------------------------------------------------------------------------------------
%	Part 2 - The Effect of Ties
%----------------------------------------------------------------------------------------

\section{Part 2 - The Effect of Ties}

Repeated Forward A* needs to break ties to decide which cell to expand next if
several cells have the same smallest f-value. It can either break ties in favor of cells with smaller g-values or in favor of
cells with larger g-values. Implement and compare both versions of Repeated Forward A* with respect to their runtime or,
equivalently, number of expanded cells.

%------------------------------------------------

% Numbered question, with subquestions in an enumerate environment
\begin{question}
	Explain your observations in detail, that is, explain what you observed and give a
reason for the observation.

\end{question}

Answer: \newline
Repeated Forward A*

\textbf{Choosing smaller g-value}
Average run time: 5.515

\textbf{Choosing larger g-value}
Average run time: 0.594 s
\par We observed that tie-breaking by the larger g-value proved to be much faster than the other one with smaller g-value. In general, when g1+h1 = g2+h2, we will choose the one with smaller h value. Here, it means the one with smaller h value will have larger g-value. The reason why we choose the one with larger g-value is that we know the g-value represents the cost of the path the agent has finished while the h-value is a prediction of the best case in the future which can potentially increase a lot. If we choose the one with larger h-value and smaller g-value, it will have a risk to increase the total cost. That's why the method of choosing the larger g-value is faster than the method of choosing the smaller g-value.


%----------------------------------------------------------------------------------------
%	Part 3 - Forward vs. Backward
%----------------------------------------------------------------------------------------

\section{Part 3 - Forward vs. Backward}

Implement and compare Repeated Forward A* and Repeated Backward A*
with respect to their runtime or, equivalently, number of expanded cells. Explain your observations in detail, that is, explain
what you observed and give a reason for the observation. Both versions of Repeated A* should break ties among cells with
the same f-value in favor of cells with larger g-values and remaining ties in an identical way, for example randomly
\newline

%------------------------------------------------

Answer: \newline
\textbf{Repeated Forward A*:}
{\obeylines\obeyspaces
\texttt{
\input{RepeatedForwardOutput.txt}
}}
\textbf{Repeated Forward A*: Average run time = 0.3089696089426676}
\newline
\textbf{Repeated Forward A*: Average expanded node = 3959.15555556}
\newline
\newline
\textbf{Repeated Backward A*:}
{\obeylines\obeyspaces
\texttt{
\input{RepeatedBackwardOutput.txt}
}}
\textbf{Repeated Backward A*: Average run time = 0.3506512123605479}
\newline
\textbf{Repeated Backward A*: Average expanded node = 176390.76087}
\par Obersvation : After comparing the run time and the number of expanded node of Repeated Forward and Backward, we can easily find that Repeated Forward A* ran faster than Backward A. But in this situation, the origin node and destination node are not the same.
\newline
By setting another python file to test, we set the origin node is same as the destination node. We use it to compare it again:
\newline
\newline
\newline
Repeated Forward:
\newline
\includegraphics[scale = 0.70]{forward.PNG}
\includegraphics[scale = 0.70]{forward2.PNG}
\newline
Repeated Backward:
\newline
\includegraphics[scale = 0.70]{backward.PNG}
\includegraphics[scale = 0.70]{backward2.PNG}
\newline
\includegraphics[scale = 0.70]{Output.PNG}
\newline
\par By comparing these two condition, we can easily say that the running time of Repeated Forward A* is obviously faster than the Repeated Backward A*.
\newline
\newline
Explanation: Here, we know the repeated backward need to expand more cells, (83718 >> 11991). During one iteration of the A* search, we start the search from a place very close to the already detected vertical obstacle, then the A* search would find the break point vertically step by step, in which case each step starts over from the beginning to the next vertical point of the obstacle. For the 

%----------------------------------------------------------------------------------------
%	Part 4 - Heuristics in the Adaptive A*
%----------------------------------------------------------------------------------------

\section{Part 4 - Heuristics in the Adaptive A*}

The project argues that “the Manhattan_distance distances are consistent in
gridworlds in which the agent can move only in the four main compass directions.” Prove that this is indeed the case.

Furthermore, it is argued that “The h-values hnew(s) ... are not only admissible but also consistent.” Prove that Adaptive A*
leaves initially consistent h-values consistent even if action costs can increase.
\newline

%------------------------------------------------

Answer: \newline
The heuristic function h is said to be consistent if 
$$ \forall(n, a, n'): h(n) \leq c(n, a, n') + h(n'),$$
where c(n, a, n') is the step cost for going from n to n' using action a.
\par In this case, the step cost for each move is 1, therefore, $$c(n, a, n') + h(n') = h(n') + 1$$ , and h(n) is either h(n') + 1 or h(n') - 1, whenever the agent makes the next move. \par In the case of h(n) = h(n') + 1, h(n) = c(n, a, n') + h(n'). \par In the case of h(n) = h(n') - 1, h(n) < c(n, a, n') + h(n').
\par Thus, the statement "the Manhattan_distance distances are consistent in grid-worlds in which the agent can move only in the four main compass directions" is true.



$$$$\par The second part asks us to prove that Adaptive A* leaves initially consistent h-values consistent even if action costs can increase. The heuristic function h is said to be consistent if 
$$ \forall(n, a, n'): h(n) \leq c(n, a, n') + h(n'),$$ where c(n, a, n') is the step cost for going from n to n' using action a.
\par For each Adaptive process, let the step cost for each move to be denoted by the variable A where A is an integer greater than 0, thus, $$c(n, a, n') + h(n') = h(n') + A$$ . 
\par As is known, for Adaptive A* search, $h_{new} = g(s_{goal}) - g(s)$, and we need to prove:
$$h_{new}(s) \leq h_{new}(s') + A,$$ 
\par which is equal to:
$$g(s_{goal}) - g(s) \leq g(s_{goa})) - g(s') + A$$
$$ => g(s') \leq (s) + A,$$ 
\par which is true.
\par Thus, the statement "Adaptive A* leaves initially consistent h-values consistent even if action costs can increase" is true.

%----------------------------------------------------------------------------------------
%	Part 5 - Heuristics in the Adaptive A*
%----------------------------------------------------------------------------------------

\section{Part 5 - Heuristics in the Adaptive A*}

Implement and compare Repeated Forward A* and Adaptive A*
with respect to their runtime. Explain your observations in detail, that is, explain what you observed and give a reason for
the observation. Both search algorithms should break ties among cells with the same f-value in favor of cells with larger
g-values and remaining ties in an identical way, for example randomly.
\newline

%------------------------------------------------

Answer: \newline

\par Repeated Forward A* = 0.35822288195292157
       Adaptive A* = 0.46544274829
       
      Here, we can say that the repeated forward A* search is slightly faster than the Adaptive A* search. The reason why cause that is the Adaptive A*
      Algorithm avoids certain path on the way and expands less cell each time, it has to traverse the whole recorded list whenever it updates its next 
      possible path. That might be the reason that Adaptive A* only save a little bit time from updating its new path.

%----------------------------------------------------------------------------------------
%	Part 6 - Memory Issues
%----------------------------------------------------------------------------------------

\section{Part 6 - Memory Issues}

You performed all experiments in gridworlds of size 101 × 101 but some real-time
computer games use maps whose number of cells is up to two orders of magnitude larger than that. It is then especially
important to limit the amount of information that is stored per cell. For example, the tree-pointers can be implemented with
only two bits per cell. Suggest additional ways to reduce the memory consumption of your implementations further. Then,
calculate the amount of memory that they need to operate on gridworlds of size 1001 × 1001 and the largest gridworld that
they can operate on within a memory limit of 4 MBytes.
\newline

%------------------------------------------------

Answer: \newline
\par The one that takes up the most memory is list. So we need to figure out a way to reduce the memory usage in our implementation. One way is to use a boolean value to check
a cell is blocked or not. And the other way is using the single list rather than two list.

\par Each object we create will just be an int type and an int type has a size of 4 bytes and we can do something similar to the class node. Let's take 4MB as 4e6 bytes, then the maximum number of grids we can have will be 4e6/4/2 = 500000. $\sqrt{500000}=707$, which means we will support a 707x707 grid-world for the worst case. For a 1001x1001 grid-world, it will take 1001*1001*4*2 = 8016008 bytes of memory which is roughly 8 MB of memory.
%----------------------------------------------------------------------------------------

\end{document}
